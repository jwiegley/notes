\documentclass{llncs}

\usepackage{makeidx}  % allows for indexgeneration
\usepackage[usenames,dvipsnames]{color}


\newif\ifdraft\draftfalse  % draft = comments and half-baked bits
% \newif\ifdraft\drafttrue  % draft = comments and half-baked bits

% Comments
\newcommand{\comment}[3]{\ifdraft\textcolor{#1}{[#2: #3]}\else\fi}
\newcommand{\fixme}[1]{\comment{red}{FIXME}{#1}}
\newcommand{\todo}[1]{\comment{red}{TO DO}{#1}}
\newcommand{\gts}[1]{\comment{OliveGreen}{GTS}{#1}}

% should be the last one
\usepackage[%
    %pdftex,%
    %pdfpagelabels,%
    %bookmarks=true,%
    % pdfdisplaydoctitle=true,%
    % pdflang=en-us,%
    % pdfencoding=auto,%
    % bookmarksnumbered=true,%
]{hyperref}

% % s/coq_eval/coqeval/
% \excludecomment{coqeval}
% % s/coq_example*/coqexamle/
% % \excludecomment{coqexample}
% \newenvironment{coqexample}%
%    {\endgraf\noindent %
%     \endgraf\verbatim}%
%    {\endverbatim}

\begin{document}

\frontmatter          % for the preliminaries

\pagestyle{headings}  % switches on printing of running heads

\mainmatter

\title{Register allocator requirements}

\author{John Wiegley \and Gregory Sullivan}

\institute{BAE Systems, Burlington MA 01803, USA}

\maketitle

\section{Register Allocation Algorithm and Correctness Properties}
\label{sec:alg}

The algorithm we implemented is fully documented in the original
paper\cite{Wimmer:2005}.  We extend that documentation by presenting several
requirements that were implicit in their presentation, but became necessarily
explicit in the Coq formalization.  Each requirement stated below was encoded
in either a type or a lemma, and proven during the course of development.

\subsection{Use positions}
\label{sec:usepos}

When code is linearized for the purpose of register allocation, each position
where a variable is used is termed a \emph{use position}.  There is also a
boolean flag to indicate whether a register is required at this use position
or not.

\paragraph{Requirements}

\begin{itemize}
\item All use positions must be given odd addresses, so that ranges do not
  extend to the next use position.
\end{itemize}

\subsection{Ranges}
\label{sec:ranges}

A list of use positions constitutes a range, from the beginning use position,
to one beyond the last.  A range may also be extended so that its beginning
and end fall outside its list of use positions.  This is done in order to
provide explicit coverage of loop bodies, to avoid reloading registers each
time the loop body is executed.

\paragraph{Requirements}

\begin{itemize}
\item The list of use positions is ordered with respect to location.
\item The list of use positions is unique with respect to location (no
  recurring use positions).
\item The beginning of a range is $\le$ its first use position location.
\item The end of a range is $>$ its last use position location.
\end{itemize}

\subsection{Intervals}
\label{sec:intervals}

An ordered list of ranges constitutes an \emph{interval}.  In the majority of
scenarios, intervals are equivalent to ranges, since only a single range is
used.  However, adding the concept of intervals allows for \emph{lifetime
  holes} between component ranges, that are otherwise allocated to the same
register.

\paragraph{Requirements}

\begin{itemize}
\item The list of ranges is ordered by the first use position in each range.
\item The list of ranges is non-overlapping.
\item The beginning of the interval corresponds to the beginning of its first
  range.
\item The end of the interval corresponds to the end of its last range.
\end{itemize}

\subsection{Scan state}
\label{sec:scanstate}

During the course of the algorithm, a position counter is iterated from the
first code position to the last.  Six separate details are managed using a
collection of lists:

\begin{enumerate}
\item A list of yet to be allocated intervals (unhandled intervals).
\item A list of active intervals (those covering the current position, and
  presently allocated).
\item A list of inactive intervals (those having a lifetime hole covering the
  current position).
\item A list of handled intervals.
\item A mapping from registers to active/inactive/handled intervals.
\item A mapping from intervals (those not unhandled) to registers.
\end{enumerate}

\paragraph{Requirements}

\begin{itemize}
\item The list of unhandled intervals is ordered by start positions.
\item The first 4 lists, taken as a set, must contain no recurring
  intervals, and have no overlapping intervals.
\item All register indices must be $<$ the maximum register index.
\item The number of intervals in the active list be $<$ the maximum register
  index.
\end{itemize}

\end{document}
